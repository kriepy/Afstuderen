% INTRODUCTION and HEALTH MONITORING
The goal of monitoring the health of people is to detect accidents or even more important to prevent accidents and critical health conditions. Changes in the daily behavior patterns can be a sign of changes in the health of people. This can be both mental or physical declines.
There are different ways to monitor the health condition of people. Cameras or microphones can be very useful to monitor peoples behavior \cite{Nagai2010204, Wu_2003_4676}, but these sensors are invading the privacy of people and often not accepted as sensors in peoples homes.\\

% PERVASIVE SENSOR SYSTEMS SUPERVISED
Simple binary sensors such us motion sensors, contact switches or pressure mats are preferable for health monitoring in home environments. These sensors are low in cost and easy to install. Moreover, they are also experienced as non-intrusive and not disturbing by the inhabitants. Numerous researchers implemented different approaches to apply activity recognition on data generated by these kind of sensors. These activities and especially changes in these activities, often referred to as ADL's, can then give valuable information on peoples health \cite{journals/hf/RogersMWF98}.
% Tapia
Tapia et al. \cite{Tapia04activityrecognition} uses a naive Bayes classifier to find activities in annotated, sensor data. They show that it is possible to find activities in ubiquitous, simple sensor data, that was obtained in real-life environments.
% Kasteren
In the work of Kasteren et al. \cite{vanKasteren:2008:AAR:1409635.1409637} two approaches for recognizing activities in sensor data are compared. The Hidden Markov Model and the Conditional Random Field are both applied to annotated, real-life sensor data. They also vary between different kind of sensor readings and show that this can improve the results for recognizing activities with their approaches.
% Wilson
Wilson et al. \cite{Wilson:2005:STA:2154273.2154280} implemented a Particle Filter to find activities in simulated as well as real-life data. They are able to distinguish the actions between multiple people in the environment.
% Hong
Hong et al. \cite{Hong2009236} uses ontologies to describe daily activities. They use an evidential network to describe activities in a hierarchical way.
\\


% UNSUPERVISED USAGE OF LDA
All of the previous approaches used annotated data. Generating this labeled data is however difficult, time consuming and the determined labels are not always accurate. For this reason supervised methods are preferred above supervised methods in this field.
Various authors applied LDA to different kind of data. This topic model is able to find abstract descriptions of activities in data automatically.\\

% Chikhaoui
Chikhaoui et al. \cite{journals/percom/ChikhaouiWP12} uses the topic model LDA in combination with sequential pattern mining to find activities in various datasets. The sequential pattern are used as words, which are the input for the LDA model.  They test their method on varied annotated data sets. The topics that are found describe activities and the accuracy is measured by comparing the topics with annotation labels. In their work the focus lays on detecting activities and not so much on the global idea of detecting behavior patterns as it is done in this work.\\

% Huynh, Casale
Huyhn et al. \cite{Huynh:2008:DAP:1409635.1409638} and Casale et al. \cite{Casale:2009} both try to discover daily routines from sensor data. Acceleration sensors that are attached to the human body generate a continuous stream of data. The data is quantized in different time intervals and in this ways artificial words are created. The dictionary, which contains all unique words, is quite big, because small variations in the sensor data generates different words. Therefore the authors cluster the data on forehand with the k-means algorithm. In this way the size of the dictionary can be reduced. The choice of $k$ however is of big influence on the outcome of the LDA model. That is why in this thesis a different approach is used to handle the big amount of variations in the artificial words.\\

% Farrahi
Farrahi et al. \cite{farrahi2008daily} applies LDA to location data gained from cell information of mobile phones. A lot of data is available and artificial words are created by combining the location of a person at three consecutive time steps and adding the time value. In this way every time of a day is described with a word. This approach of dividing a day into words is also adopted in this thesis. In Farrahi's work only 512 different words are possible, but about 2800 days of 68 different people are available. This ratio of vocabulary size and available data makes their approach a successful way to find latent topics in location data, without the need of clustering the data on forehand. Their way of describing the features is a good option to capture transitions in locations.\\

% Castanedo
In the work of Castanedo et al. \cite{EXSY:EXSY12033} they also apply the LDA model on sensor data without pre-clustering the dictionary. For their work much data was available, which was obtained in an office environment. The words are however represented differently than it is done in the work of Farrahi et al. Not every time of a day is divided into words, but only time periods that contain sensor activations build the artificial words. They indicate that it can become difficult to give a good interpretation of the topics, that are found. So it is questionable if leaving out time periods without sensor activations in the feature representation leads to a better result.\\





