% INTRODUCTION and HEALTH MONITORING
The goal of monitoring the health of people is to detect accidents or even more important prevent accidents and critical health conditions. Changes in the daily behavior patterns can be a sign of changes in the health of people. This can be both mental or physical declines.
There are different ways to monitor the health condition of people. Cameras or microphones can be very useful to monitor peoples behavior \cite{Nagai2010204, Wu_2003_4676}. But they are invading the privacy of people and often not accepted as sensors in peoples homes.\\

% PERVASIVE SENSOR SYSTEMS SUPERVISED
Simple binary sensors such us motion sensors, contact switches or pressure mats are preferable for health monitoring in home environments. These sensors are low in cost and easy to install. They are also often experienced as non-intrusive and not disturbing by the inhabitants. Numerous researchers implemented different approaches to apply activity recognition on this kind of data. These activities and especially changes in these activities, often referred to as ADL's, can then give valuable information on peoples health \cite{journals/hf/RogersMWF98}.\\
% Tapia
Tapia et al. \cite{Tapia04activityrecognition} uses a naive Bayes classifier to find activities in annotated, sensor data. They show that it is possible to find activities in ubiquitous, simple sensor data, that was obtained in real-life environments.\\
% Kasteren
In the work of Kasteren et al. \cite{} two approaches for recognizing activities in sensor data are compared. The Hidden Markov Model and the Conditional Random Field are both applied to annotated, real-life sensor data. They also vary between different kind of sensor readings and show that this can improve the results for recognizing activities with their approaches.\\
% Wilson
Wilson et al. \cite{Wilson:2005:STA:2154273.2154280} implemented a Particle Filter to find activities in simulated as well as real-life data. They are able to distinguish the actions between multiple people in the environment.\\
% Hong
Hong et al. \cite{Hong2009236} uses ontologies to describe daily activities. They use an evidential network to describe activities in a hierarchical way.
\\


% UNSUPERVISED USAGE OF LDA
In the previous approaches annotated data is used to find activities in sensor data. Generating labeled data is however difficult. It is time-consuming and the labels can be inaccurate. Variations in the way people behave cannot be captured. That is why unsupervised methods to find activities or behavior patterns in sensor data is more convenient that supervised methods. Various authors applied LDA to different kind of data. This topic model is able to find abstract descriptions of activities in data automatically.\\
% Chikhaoui
Chikhaoui et al. \cite{journals/percom/ChikhaouiWP12} uses the topic model LDA in combination with sequential pattern mining to find activities in various datasets. The sequential pattern are used as words that are needed as input for the LDA model. In their work they focus on detecting activities and not so much on daily behavior patterns of people.\\
% Huynh, Casale
Huyhn et al. \cite{} and Casale et al. \cite{} both apply LDA to sensor data obtained from wearable sensors. From acceleration features, that are clustered in advance, they generate the artificial words. The clustering is necessary to group similar words together. In this way the size of the dictionary is reduced and LDA can find meaningful topics in the data. Choosing the amount of clusters on forehand is however difficult and has a big influence on the outcome of the LDA model.\\
% Phung DOEN
Phung et al. \cite{conf/percom/PhungATVK09} applys LDA to data that is gained of a WiFi network. They find behavior patterns of people in their work environment.\\
% Farrahi DUIDELIJKER
Farrahi et al. \cite{farrahi2008daily} applys LDA to location data gained from cell information of mobile phones. A lot of data is available and the simple location description that is used to create the artificial words make it possible to apply the BOW representation of the data directly. No clustering of the data is necessary.\\
% Castanedo
In the work of Castanedo et al. \cite{EXSY:EXSY12033} they apply the LDA model to a data set collected from a sensor network in an office environment. The big amount of sensors installed in the test environment and big amount of data collected also makes it possible to apply the BOW model to this data set directly. However they have difficulties to give an interpretation of the detected topics.\\





