Monitoring the health of elderly can be done in many different ways. Also the goal of monitoring elderly differs. One goal can be detecting accidents where in other systems the focus lays on the long-term health monitoring, which is in fact a way to prevent accidents. In \cite{Tamura1998573} for example a system is described that is installed at peoples home and monitors their health. They use a variety of sensors and the system can give a detailed information of the health condition of the inhabitant. Sensors were added in bathtub, toilet and bed. These devices where specially build for the experiments which makes this approach quite expensive.\\
% In \cite{Nagai2010204} a system is describes that uses cameras in the homes of people to monitor their health condition. If an emergency occurs staff can handle appropriate. Also family members have access to the video streams and can report accidents.
In \cite{Kwon20125774} more simple sensors are used to monitor the health of solitary living people. Different sensors in the houses are tracking the health condition and the system is able to automatically contact emergency services or give the signal to send a caretaker at the home of the subject. A combination of static and wearable sensors is used, some of these sensor record automatically and other needs be triggered by the inhabitant.\\
Systems that monitor the health of people need to be not intrusive, which means that the choice of sensors becomes important. People do not want to be watched all the time, which makes the use of camera or microphones (\cite{Nagai2010204}, \cite{Wu_2003_4676}) mostly not eligible.
It is also desirable that sensors are low in cost. Another thing that is desirable is that the system collects data automatically. Inhabitants do not want to need to interact with the system constantly and the system should not affect their daily living. To meet these conditions several researchers focus on the task of activity recognition with simple sensors.\\

In \cite{Hong2009236} a system to recognize activities is introduced that is based on ontolgies. Simple binary sensors, like movement sensors, contact switches and pressure mats, are used. Activities are described with ontolgies that are build from the sensor data.
In \cite{Tapia04activityrecognition} contact switches that also generate binary data are used. A naive Baysian Network is trained with annotated data. This data is is collected from the inhabitants of two different houses.
In the work of Wilson et al. \cite{Wilson:2005:STA:2154273.2154280} a particle filter is applied to binary sensor data. They focus on handeling multiple inhabitants in house.\\
%'Huynh' applies LDA to a wearable sensor

All the work that is described above about activity recognition uses a supervised manner to detect activities in the data. Annotating the data has some disadvantages as it is pointed out in the work of Chikhaoui et al. \cite{journals/percom/ChikhaouiWP12}. In their work they use the topic model LDA in combination with sequential pattern mining. These sequential patterns build the words, that are needed as input for the LDA model. They apply their model to different data sets with different sensor types. The focus in this work lays more on recognizing activities than finding behavior pattern on a day as it is done in this thesis.\\
In the work of Castanedo et al. \cite{EXSY:EXSY12033} an overview of different applications of the LDA model is given. They apply the LDA model to a data set collected from a sensor network in an office environment. The big amount of sensors installed in the test environment and big amount of data collected makes it possible to apply the Bag-of-words model to this data set. However they have difficulties to give an interpretation of the detected topics.\\
In the work of Casale et al. \cite{Casale:2009} LDA is apllied to data coming from a wearable sensor. They reduce the size of the dictionary by clustering the sensor readings. The size of the clusters is difficult to choose, which is also shown in this thesis.
A different kind of data is used in the work of Farrahi et al. \cite{farrahi2008daily}. Cell information of mobile phones give information about the location of a person. With LDA they are able to find different kind of behaviors of people. They use a nice feature representation with which transitions of people can be modelled.