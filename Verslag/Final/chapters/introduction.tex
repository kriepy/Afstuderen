%NOG DOEN:
% 

%START
The life expectancy of people is assumed to rise continuously in the following years \cite{4864}. As a consequence the percentage of elderly increases. Elderly people often need more health care but studies show that they prefer to live at home \cite{Cavallo:1315167}. The manpower to take care of elderly people is not always available. That is why monitoring the health condition of people in their home environment becomes more and more important. In this way slowly emerging declines in the health condition of people can be detected and appropriate health care can be granted if it is necessary before critical health conditions are reached.\\

%Monitoring elderly
New systems give the possibility to monitor elderly from the distance or even automatically \cite{Tamura1998573}. Different systems give the possibility to detect accidents or monitor the health condition on a long time basis. Some systems still need the action of a person involving into the system. For example the inhabitant has to push a button if an accident occurs \cite{Kwon20125774}. Other systems use cameras to monitor the elderly \cite{Nagai2010204,Mubashir2013144}. But these methods are privacy-sensitive and often not adopted by the elderly.\\

%Pervasive sensors
Less intrusive methods use simple sensors like motion sensors or pressure mats that are placed in the homes of people \cite{Tapia04activityrecognition,4912776}. Extracting valuable information out of this data is however difficult and that is why activity recognition is often done to make the data more easily to interpret. Different techniques attempt to extract 'Activities of daily living' (ADL's). Changes in the ADL's can be a sign for declines of peoples health \cite{Tapia04activityrecognition}.\\
Some researchers depend on annotated data to find ADL's in sensor data \cite{Tapia04activityrecognition,Hong2009236,Wilson:2005:STA:2154273.2154280}. But the task of labeling the data is time consuming and also intrusive, if the labeling is done by a different person. The knowledge of being observed by a sensor, for example a camera, might change the behavior patterns of a person and in this way the data is not accurate. Annotation that is done by the subject himself also might not by accurate, because the person has to interrupt his daily behavior to write down the annotation of the activities.\\

% unsupervised topic models on sensor data, wat is al gedaan
A way to automatically find behavior pattern in sensor data is by applying a topic model to the data. Topic models are initially designed for classifying text documents and they are able to find abstract topics, such as 'politics', 'sports', 'finances' etc. But a number of researchers have applied the topic model 'Latent Dirichlet Allocation' (LDA) to different kind of sensor data \cite{farrahi2008daily,journals/percom/ChikhaouiWP12}. The found topics are comparable to ADL's and can give a good representation of peoples daily behavior.\\
There is however a mayor difference between textual and sensor data. In the LDA model presented by Blei \cite{blei2003latent} the documents are represented by a Bag-of-words (BOW) model. A BOW model is an unordered representation of words, that does not take the grammar or the order of the words into account.\\
To make use of the LDA model on sensor data one has to create artificial words. But sensor data is mostly time dependent and that is why in most approaches a time value is added to the artificial words. There are numerous ways ways to create words from sensor data. \\
Some researchers create artificial words by simply adding a time-value to the the sensor data and directly apply the BOW model to these 'words' \cite{farrahi2008daily,EXSY:EXSY12033}. This approach requires a large dictionary of words to find behavior patterns in the data, to be able to capture all the variations in artificial words. For this reason other researchers first cluster the data and then apply the LDA model \cite{Huynh:2008:DAP:1409635.1409638,Casale:2009}. In this way the size of the dictionary is smaller and less data is required.  But finding the correct clusters is however difficult.\\

% mijn bijdrage
In this thesis the two new topic models 'LDA-Gaussian' and 'LDA-Poisson' are developed. These models are able to capture similar observations/words into the same topic automatically. The clustering and the LDA model are combined into one model and except for the construction of the artificial words, no further pre-processing is necessary .\\
In the original LDA model (Blei) the topics are described by a multinomial distribution over the words of the vocabulary. In the 'LDA-Gaussian' model this distribution is replaced with a Gaussian distribution, so that similar words are caught in the same topic. `LDA-Poisson' is a variation on this model, where the underlying distribution is Poisson. In this way event based sensor data can be modeled.\\
The parameters of both models are found with an EM-procedure, which uses the likelihood of the model to converge to the optimal model parameters. The models are applied on real sensor data, that is obtained from the houses of solitary living elderly. 
All sensors are binary and are experienced as non-intrusive by the inhabitants. These elderly people live in a care home and get health care on a regular basis. Still they are able to live on their own.\\


% START outline verslag
In the next chapter of this thesis an overview of related approaches are given. In chapter \ref{chapter:data_description} the data that is used is described in more detail and after that representation of the features is given in \ref{chapter:features}. This chapter is followed by chapter \ref{chapter:topic_models} which introduces the 'LDA-Gaussian' and 'LDA-Poisson' models. Chapter \ref{chapter:experiments} contains the different experiments that are performed on the available data. In chapter \ref{chapter:future_work} suggestions for future work are given and the conclusion of the thesis is presented in chapter \ref{chapter:conclusions}.


