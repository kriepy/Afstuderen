%NOG DOEN:
% bij monitoring elderly, Related work begin deel invoegen

%START
The life expectancy of people is assumed to rise continuously \cite{4864}. That means that also the percentage of elderly increases. Elderly people often need more health care and studies show that they like to live at home \cite{Cavallo:1315167} . That is why monitoring the health condition of people in their home environment becomes more and more important.\\

%Monitoring elderly
New techniques give the possibility to monitor elderly from the distance or even automatically. These systems give the possibility to detect accidents, if for example a person falls on the ground \cite{Mubashir2013144}. Other applications are monitoring the health condition on a long term basis(find cite). Some of these techniques use cameras that are placed in the homes of elderly \cite{Nagai2010204}. But these methods are privacy-sensitive and often not adopted by the elderly(cite).\\
Less intrusive methods use simple sensors like motion sensors or pressure mats that are placed in the homes of people [Tapia]. Reading and understanding this sensor data is difficult and that is why activity recognition is often done to make the data more easily to interpret. Different techniques attempt to extract 'Activities of daily living' (ADL's). Changes in the ADL's can be a sign for declines of peoples health. (cite volgens mijn van tapia)\\
Some researchers depend on annotated data to find ADL's in sensor data [Tapia,Ontologies,Particle Filter]. But the task of labeling the data is time consuming and also might effect the output of the sensor data. If for example a person is asked to notate his activities during the day, the action of annotating also takes time and the sensor data might be inaccurate. A camera that is placed in the home environment might also affect the way people behave, due to the fact that they know a camera is watching them, and with that it affects the data that is collected. That is why unsupervised methods are more preferable.\\


% Wat is er tot nu toe gedaan met sensor data en topic modelen?
Topic models seem to be a good way to find behavior patterns in data automatically without the need of annotation. Different authors apply the Topic model 'Latent Dirirchlet Allocation' (LDA) [Blei] to different kind of sensor data[cite]. The latent topics that are found with this model are an abstract description of an activity which are in some cases related to ADL's. The topics and the distribution of the topics can be a good representation of peoples daily behavior patterns.\\
LDA is initially developed for clustering of text documents and in the algorithm introduced by Blei [] the text documents are represented by a Bag-of-Words model (BOW), which is a unordered collection of words for every document, that does not take the grammar or the order of the words into account.\\
Creating artificial words from sensor data can be done in different ways. Sensor data is time dependent and that is why authors group  observations together in a time interval and the time value is added. In this way the order of the observations is in fact taken into account [farahhi, castenado]. \\
It is often the case that not a lot data is available. Still there exists a lot of information of the data. The BOW cannot catch this information because small variations in the sensor data lead to different words and the correlations of these words is thrown away when the BOW model is applied. That is why some author cluster the found words by forehand, which lead to a smaller dictionary of observations.\\
Others extract the sequential patterns of the sensor data. These sequences then are building the words of the dictionary.

Verschillende aanpakken zijn gebruikt om uit sensor data behavior patronen te vinden en de gevonden topics are comparable to ADL's. LDA is often used as a basis [Blei]. This model searches Latent topics in data, but this model is in eerste instantie devolped for clustering van text documenten. The Bag-of-words model (BOW), which is the basis for LDA, is not that easily appliable on sensor data. In the BOW model a dictionary is build of all possible words, which can be seen as independent dimensions. So for text document the data space that is used is a high dimensional binary space. Sensor data contains darentegen  vaak not a lot of dimension, but the dimension is mostly not binary and the observation per dimension are more variate. Als is sensor data often depending on the time and there with the volgorde van eventen is also very important.\\
Different researchers apply the BOW to sensor data by discretizing the data in some way and are building virtual words of the observations. Often a way of preprocessing the data is neccessary to find the Vocabulary and apply the LDA model. In [Huynh,Cavallo] the words that are extracted from the sensor data are clustered with k-means to group variations in the sensor together and in this way reduce the size of the dictonary. In [Chikhaoui] sequential patterns are found in the data which are used as words in the dictonary. 
 is here used for describing different text documents. Sensor data is not so easily described with this method, due to the time dependency of the data and the small amount of dimensions which are though more variate in the values. 
With an unsupervised method labeling the data is not necessary. And systems can record for a long time and generate a lot of data. Verifying a method is in this way much harder but the patterns that are found with theses methods might give interesting results that can be used by care takers and health professionals.\\

% Wat is mijn bijdrage en toevoegingen voor deze modellen?
In this thesis an algorithm is developed that does not need preprocessing of the data but can handle variations in the different dimensions of the data. The clustering is added into the LDA model itself. And a topic will be described with a set of parameters for a Gaussian distribution in every dimension. A variation of the algorithm is given where every dimension is described with a Poisson distribution. The algorithms are applied to real-life sensor data, that is gathered in the houses of solitary living elderly. The latent topics that are found describe the sensor data of a day, where a topic can be seen as an abstract description of an activity or a combination of activities. Example of such a topics are 'preparing breakfast' or 'going to toilet'.\\

%waar komen topic modellen vandaan?
Topic models are often used in the field of classifying text documents. The idea is that every document might belong to a couple of different topics and a topic can be described with a distribution over words. A newly seen document can then be assigned to some topics according to the words that occur in the document. In this way a document is described with a distribution of topics. A detailed description of the method and how the model parameters are found are described in the work of Blei et al. \cite{blei2003latent}. This method does not need any annotation of the data and finds the topics automatically.

%Hoe verschilt ons data van text documenten?
To make use of this method a suitable feature representation of the available data is required. The features are found by counting the activations of sensor groups. In this way a relative low number of dimensions are found, which can contain positive integers. In theory these numbers have an infinite range, but in practice the count of sensor activations of every dimension is reduced to a maximum.\\
This feature representation differs a lot to the Bag-of-words model that is often used for the description of textual data. In this representation a dictionary of all possible words is build and a document can be described with a binary vector of the same length as the size of the dictionary. With respect to our data, text data thus has much more dimension but every dimension only contains a binary value.

%wat zijn de verschillende aanpakken die we hebben gebruikt?
Due to the feature representation of the sensor data the Bag-of-words model is not suitable. Therefore a model based on LDA is build that models every dimension with a Gaussian distribution. In this way small variations of the sensor data are captured. A variation of this model is build where the underlying distribution of the dimensions becomes the Poisson distribution. This suits the way the data is described better, because every dimension of the features contains a count of events. The parameters of both models are found with an EM-procedure, which uses the likelihood of the model to converge to the optimal model parameters.\\


% START outline verslag
In the next chapter of this thesis an overview of related approaches are given. In section \ref{sec:DataDesc} the data that is used is described in more detail and after that representation of the features is described in \ref{sec:features}. This is followed by section \ref{sec:TopicModels} which describes the different ways of the usage of Topic Models with the available data. The section \ref{sec:Experiments} contains the different experiments that are performed. Section \ref{sec:Disc} the outcome of the experiments are discussed and finaly the conclusion of the report is given in section \ref{sec:Conc}.