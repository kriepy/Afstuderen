In this thesis behavior patterns of people are analyzed with Topic models. Two novel variations of the Latent Dirichlet Allocation (LDA) model are presented. The models give the opportunity to detect patterns in low-dimensional sensor data in an unsupervised manner. LDA-Gaussian, the first variation of the model, is a combination of a Gaussian Mixture Model and the LDA model. Here the multinomial distribution of the topics, that is normally used in the LDA model, is replaced by a set of Gaussian Distributions. In this way similar looking sensor data is automatically grouped together and captured in the same topic.
LDA-Poisson, the second variation of the model, takes a set of Poisson Distribution for the topic descriptions. This model is more suited to handle counts of stochastic events. The parameters of both models are determined with an EM-algorithm.
Both models are applied to real sensor data, which is gathered in the homes of elderly people. It is shown that meaningful topics can be found and that a semantic description of these topics can be given.