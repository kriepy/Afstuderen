\documentclass[11pt,a4paper]{article}
\title{Behaviour Analysis of Elderly using topic models}
\author{Kristin Rieping}
\usepackage[latin1]{inputenc}
\usepackage{amsmath}
\usepackage{amsfonts}
\usepackage{amssymb}
\begin{document}

\section{Introduction}
The world population increases unceasingly and thereby the percentage of elderly also increases. The manpower to take care of elderly is diminishing. So it becomes more and more important to give elderly the opportunity to live on there own and be more independent of health care.
New techniques give the possibility to monitor elderly from the distance or even automatically. Some of these techniques use cameras that are placed in the homes of elderly. But these methods are privacy-sensitive and often not adopted by the elderlies. []
Other methods use motion sensors that are placed at different locations in the homes of elderly. 
Reading and interpreting this sensor data is often difficult and that is why often activity recognition is done to make the data more readable. But different approaches show that this is also a though challenge. [ ]
Machine Learning techniques often require a lot of annotated data, but the task of labeling a data set is time consuming and and can also influence the output of the data while doing so.\\
Therefore we focus in this thesis on finding features in an unsupervised manner. These features should be meaningful and easy to interpret. We do not focus so much on activity recognition but look more widely on the daily behavior of a person given sensor data. As an inspiration we use the analysis of human routines described in [Farrahi]. With an EM-algorithm for Latent Dirichlet Allocation (LDA) we try to distinguish topics in sensor data. The topics should describe different behaviors in the daily routines of a person. The LDA algorithm described in [Blei] is adjusted to fit the given data.







\end{document}